% !TeX spellcheck = de_DE
\section{Gesamtwertung der Veranstaltungen}
Die Einzelwertungen der Veranstaltungen werden zu einer Gesamtwertung für jeden Starter sowie für jedes Team zusammengerechnet. Wer die meisten Punkte hat, ist Sieger. Nur Sportler die sich zu Saisonbeginn sammelgemeldet haben, werden gewertet. Eine Ausnahme besteht für Sportler, die im Laufe der Saison eine neue Lizenz erwerben.

Die Gesamtwertung für jeden Starter erfolgt getrennt nach Damen und Herren:
\begin{itemize}
	\item BIC Gesamtwertung Overall
	\item BIC Gesamtwertung Overall U23 (17 bis 22 Jahre)
	\item BIC Altersklassenwertung:
	\begin{itemize}
		\item Junioren
		\item Aktive
		\item Senioren 35
		\item Senioren 40
		\item Senioren 50
		\item Senioren 60
		\item Senioren 70
	\end{itemize}
	\item KIC Altersklassenwertung:
	\begin{itemize}
		\item Schüler D
		\item Schüler C
		\item Schüler B
		\item Schüler A
		\item Cadetten
		\item Jugend
	\end{itemize}
	\item SIC
\end{itemize}

Die Gesamtwertung für jedes Team:
\begin{itemize}
	\item Teamwertung Speed
	\item Teamwertung Nachwuchs
	\item Teamwertung Gesamt
\end{itemize}

\subsection{Gesamtwertung für jeden Starter}

Die Einzelwertungen werden in einer Gesamtwertung zusammengerechnet, wenn die minimal notwendige Anzahl von Ergebnissen bei Veranstaltungen erreicht wurde. Es wird nur eine maximale Anzahl der besten Ergebnisse in die Gesamtwertung berücksichtigt. Die restlichen Ergebnisse sind Streichresultate. Die maximale Anzahl der gewerteten Ergebnisse und die Streichresultate sind abhängig von der Anzahl der angebotenen Veranstaltungen (siehe Anhang~\ref{sec:appendix-notwendige-ergebnisse}).

\subsubsection{BIC Gesamtwertung Overall}
\label{subsec:bic-gesamtwertung-overall}
Es wird eine bestimmte Anzahl der besten Ergebnisse der BIC-Einzelwertungen zusammengerechnet, wenn minimal notwendige Anzahl von Ergebnissen erreicht wurde (Siehe Anhang~\ref{sec:appendix-notwendige-ergebnisse}).

\subsubsection{BIC Gesamtwertung Overall U23}
\label{subsec:bic-gesamtwertung-u23}
Wie~\ref{subsec:bic-gesamtwertung-overall}, jedoch nur für Starter, die 17 bis 22 Jahre alt sind. Für die Einteilung in diese Altersklasse ist jeweils das am 31. Dezember des laufenden Jahres erreichte Alter maßgeblich.

\subsubsection{BIC Altersklassenwertung}
\label{subsec:bic-ak-gesamtwertung}
Wie~\ref{subsec:bic-gesamtwertung-overall}, aber pro Altersklasse.

\subsubsection{KIC Altersklassenwertung}
\label{subsec:kic-gesamtwertung}
Es wird eine bestimmte Anzahl der besten Ergebnisse der KIC-Einzelwertungen zusammengerechnet, wenn minimal notwendige Anzahl von Ergebnissen erreicht wurde (siehe Anhang~\ref{sec:appendix-notwendige-ergebnisse}).

\subsubsection{SIC Gesamtwertung}
\label{subsec:sic-gesamtwertung}
Es wird eine bestimmte Anzahl der besten Ergebnisse der SIC-Einzelwertungen zusammengerechnet, wenn minimal notwendige Anzahl von Ergebnissen erreicht wurde (siehe Anhang~\ref{sec:appendix-notwendige-ergebnisse}).

\subsection{Teamwertungen}
\subsubsection{Teamwertung Gesamt}
\label{subsec:teamwertung-gesamt}
Die Teamwertung Gesamt wird aus den Punkten der BIC Veranstaltungen und der KIC Veranstaltungen für alle Sportler eines Teams ermittelt, zusätzlich kommen noch die erzielten Punkte der drei besten Sportler aus der SIC Gesamtwertung eines Teams und die Summe der Punkte der drei besten Mannschaften aus den BIC Teamwertungen hinzu (d.h. es werden pro BIC Teamveranstaltung die besten drei besten Mannschaften gewertet und aufsummiert). Es wird nur die in Anhang~\ref{sec:appendix-notwendige-ergebnisse} festgelegte Anzahl der besten Ergebnisse der jeweiligen Veranstaltungen zusammengerechnet, jedoch werden die Sportler abweichend zu den restlichen Wertungen schon ab einem Ergebnis berücksichtigt. Das Team mit den meisten Punkten ist Sieger der Teamwertung Gesamt.

\subsubsection{Teamwertung Speed}
\label{subsec:teamwertung-speed}
Die Teamwertung Speed wird in mehreren Schritten berechnet.

Als erstes wird die Summe der Einzelpunkte der besten drei Sportler (geschlechtsunabhängig) eines Teams für jede BIC Veranstaltung pro Team aufsummiert. Die Teams werden nach deren Summe von der größten zur kleinsten sortiert, wobei die größten Summe die beste Platzierung in der Wertung ergibt (= Platz 1). Die Teams erhalten die Platzierungspunkte nach Anhang~\ref{sec:appendix-punkte-nach-platz}.

Als zweites werden die Punkte des besten Sportlers (geschlechtsunabhängig) eines Teams für jede SIC Veranstaltung pro Team aufsummiert. Die Teams werden nach deren Summe sortiert und erhalten die Platzierungspunkte nach Anhang~\ref{sec:appendix-punkte-nach-platz}, wie im ersten Schritt.

Als drittes werden die Punkte der besten Mannschaft einer BIC Team Veranstaltung aufsummiert. Die Teams werden nach deren Summe sortiert und erhalten die Platzierungspunkte nach Anhang~\ref{sec:appendix-punkte-nach-platz}, wie im ersten Schritt.

Als viertes werden die Platzierungspunkte der Schritte 1 bis 3 für jedes Team addiert. Das Team mit den meisten Punkten hat gewonnen.

Um als Team in der Teamwertung Speed gewertet zu werden muss eine Gebühr von \EUR{50} an den Verband entrichtet werden. Teams, die nicht gewertet werden, werden erst nach der Addition der 3 Einzelschritte aus der Gesamtwertung entfernt.

\subsubsection{Teamwertung Nachwuchs}
\label{subsec:teamwertung-nachwuchs}
Die Punkte der KIC Altersklassenwertung von den besten drei Sportlern eines Teams werden addiert. 

Jedes Team hat die Möglichkeit bis zu 2 Teams in der Teamwertung Nachwuchs werten zu lassen.
