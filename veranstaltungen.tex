% !TeX spellcheck = de_DE
\section{Veranstaltungen}
Der Ausrichter einer Veranstaltung hat eine ausführliche Ausschreibung zu erstellen (siehe Anhang~\ref{sec:appendix-inhalt-ausschreibung}). Diese Ausschreibung ist vom Vorsitzenden der Kommission Inline-, Fitness- und Speedskating im Bayerischen Rollsport- und Inlineverbandes (BRIV) zu genehmigen.

Die Inline Cup Serie des Bayerischen Rollsport- und Inlineverbandes (BRIV) gliedert sich wie folgende Kategorien:

\begin{description}
	\item[Bayern Inline Cups (BIC)] \hfill \\
	Der BIC ist eine jährliche Serie für die Altersklassen ab Junioren bis Senioren 70 mit minimal vier und maximal zehn Veranstaltungen.
	\item[Kids Inline Cups (KIC)] \hfill \\
	Der KIC ist eine jährliche Serie für die Altersklassen ab Schüler D bis Jugend mit minimal vier und maximal zehn Veranstaltungen.
	\item[Sprint Inline Cups (SIC)] \hfill \\
	Der SIC ist eine jährliche Serie mit minimal vier und maximal zehn Veranstaltungen.
\end{description}

\subsection{BIC Veranstaltung}
Die Rennen des BIC werden größtenteils auf der Straße ausgetragen. Folgende Strecken sind möglich:
\label{subsec:bic-veranstaltung}
\begin{itemize}
	\item Shortsprint (bis 500m)
	\item Bergzeitfahren (Mindestlänge 2 km) oder
	\item Einzelzeitfahren (Mindestlänge 3 km) oder
	\item Massenstartrennen in jeder Länge zwischen 5 km und 50 km
\end{itemize}

\subsubsection{Veranstaltung BIC Altersklassenwertung (BIC AK)}
Wie~\ref{subsec:bic-veranstaltung}

\subsubsection{Veranstaltung BIC Teamwertung (BIC Team)}
Für die BIC Teamwertung (BIC Team) werden zusätzliche Rennen angeboten. Folgende Rennen sind möglich:
\begin{itemize}
	\item Teamstaffeln
	\item Teamverfolgung
	\item Teamzeitfahren
	\item Teamsprint
\end{itemize}
Nähere Einzelheiten ergeben sich aus der jeweiligen Ausschreibung. Pro Team dürfen maximal fünf Mannschaften an den Start gehen. Alle teilnehmenden Sportler müssen mindestens in der Altersklasse Schüler A oder älter startberechtigt sein.

\subsection{KIC Veranstaltung}
Die Rennen des KIC werden in der Halle, auf der Bahn und auf der Straße ausgetragen. Die Obergrenze der Streckenlänge ergibt sich aus der maximal zulässigen Streckenlänge für die jeweilige Altersklasse (siehe Anhang~\ref{sec:appendix-altersklassen}). Eine Veranstaltung für den KIC muss mindestens über eine Strecke für jede Altersklasse durchführen. Es sollten soweit möglich immer Geschicklichkeitsläufe angeboten werden. Falls KIC Veranstaltungen zusammen mit einem BIC Wettkampf ausgetragen werden, so werden die Sportler des KICs in ihren Altersklassen jeweils ab Platz 1 unabhängig von der Platzierung im eigentlichen Wettkampf platziert.

\subsection{SIC Veranstaltung}
Eine Veranstaltung für den SIC findet immer über eine Strecke zwischen minimal 30m und maximal 1.000m statt. Es ist auch möglich, dass an einer Veranstaltung eine bestimmte Strecke mehrfach gelaufen und über die Addition der Platzierung bestimmt wird. Der Austragungsmodus ist der jeweiligen Ausschreibung zu entnehmen.
