% !TeX spellcheck = de_DE
\section{Inhalt der Ausschreibung einer Veranstaltung}
\label{sec:appendix-inhalt-ausschreibung}
Eine Ausschreibung einer Veranstaltung im Rahmen des Bayern Inline Cups, des Kids Inline Cups und/oder des Sprint Inline Cups muss ergänzend zu den Anforderungen der WKO 9.5 „Ausschreibung“ noch Antworten für die folgenden Fragen enthalten:

\begin{itemize}
	\item In welche Wertung (BIC, KIC, SIC) fließen die Gesamtwertungen der Veranstaltung ein
	\item Wie wird die Gesamtwertung ermittelt. Für die BIC Veranstaltungen wird das auf der ersten SK Sitzung im laufenden Jahr festgelegt.
	\item Wie werden Punktgleichheit / Zeitgleichheit behandelt.
\end{itemize}


\section{Altersklassen (laut WKO)}
\label{sec:appendix-altersklassen}
Für die Einteilung in eine Altersklasse ist jeweils das am 31. Dezember des laufenden Jahres erreichte Alter maßgeblich.

\begin{center}
\begin{tabular}{|l|l|l|}
	\hline
	\thead{Altersklasse} &
	\thead{Alter} &
	\thead{Maximale zulässige Strecke} \\ \hline
	Schüler D      & 5 bis 6 Jahre & 500m \\ \hline
	Schüler C      & 7 bis 8 Jahre & 1.000 m \\ \hline
	Schüler B      & 9 bis 10 Jahre & 2.000 m \\ \hline
	Schüler A      & 11 bis 12 Jahre & 3.000m \\ \hline
	Kadetten       & 13 bis 14 Jahre & 10.000m \\ \hline
	Jugend         & 15 bis 16 Jahre & Halbmarathon \\ \hline
	Junioren       & 17 bis 18 Jahre & Marathon \\ \hline
	Aktive         & ab 19 Jahre & keine Einschränkung \\ \hline
	Masters        & ab 35 Jahre & keine Einschränkung \\ \hline
\end{tabular}
\end{center}

\textit{Siehe WKO Punkt 7 Wettkampfklassen und Punkt 8.2 Maximal zulässige Streckenlängen.}

\section{Rollengrößen (laut WKO)}
\label{sec:appendix-rollengroessen}

\begin{center}
\begin{tabular}{|l|l|l|}
	\hline
	\thead{Altersklasse} &
	\thead{Bahn} &
	\thead{Straße} \\ \hline
	Schüler D & 80mm & 80mm \\ \hline
	Schüler C & 84mm & 84mm \\ \hline
	Schüler B & 90mm & 90mm \\ \hline
	Schüler A & 90mm & 90mm \\ \hline
	Kadetten & 100mm & 100mm \\ \hline
	Jugend & 110mm & 110mm \\ \hline
	Junioren & 110mm & 125mm \\ \hline
	Aktive & 110mm & 125mm \\ \hline
	Masters & 110mm & 125mm \\ \hline
\end{tabular}
\end{center}

\textit{Siehe WKO Anlage ``Maximale Rollengrößen''}

\section{Punktevergabe nach Zeit}
\label{sec:appendix-punkte-nach-zeit}
Die Punkte werden für jeden Sportler, der Ziel erreicht hat und vollständige Distanz gefahren ist (Finisher), wie folgt berechnet:

Punkte Einzelveranstaltung = ((100 * Sieger Zeit in Sek.) / Sportler Zeit in Sek.) (+ Punkte nach Tabelle Bonuspunkte)

\begin{center}
\begin{tabular}{|l|r|}
	\hline
	\thead{Platz} &
	\thead{Punkte} \\ \hline
 1 & 3,000 \\ \hline
 2 & 2,500 \\ \hline
 3 & 2,000 \\ \hline
 4 & 1,750 \\ \hline
 5 & 1,500 \\ \hline
 6 & 1,250 \\ \hline
 7 & 1,000 \\ \hline
 8 & 0,750 \\ \hline
 9 & 0,500 \\ \hline
	10 & 0,250 \\ \hline
\end{tabular}
\end{center}

Für grundsätzlich alle BIC Veranstaltungen werden die Punkte nach Zeit zuzüglich möglicher Bonuspunkte vergeben. BIC Veranstaltungen, die anders (Punkte nach Platzierung) gewertet werden, sind in Anhang~\ref{sec:appendix-veranstaltungen-nach-platzierung} zu finden.

\section{Punktevergabe nach Platzierung}
\label{sec:appendix-punkte-nach-platz}
Jeder platzierte Sportler bekommt Punkte gemäß der nachfolgenden Tabelle. Die Platzierung ergibt sich aus der Reihenfolge der Zielankunft bzw. des Ausscheidens in einem Massenstartrennen oder die Zeit bzw. Zeitaddition in einem Einzelstartrennen.

\begin{center}
\begin{tabular}{|l|r|r|}
	\hline
	\thead{Platz} &
	\thead{Punkte} &
	\thead{Abstand pro Platz} \\ \hline
	Platz 1       & 100,000 & 0,000 \\ \hline
	Platz 2       &  99,000 & 1,000 \\ \hline
	ab Platz 3    &  98,500 & 0,500 \\ \hline
	ab Platz 11   &  94,750 & 0,250 \\ \hline
	ab Platz 21   &  92,300 & 0,200 \\ \hline
	ab Platz 101  &  76,350 & 0,150 \\ \hline
	ab Platz 201  &  61,400 & 0,100 \\ \hline
	ab Platz 401  &  41,450 & 0,050 \\ \hline
	ab Platz 601  &  31,475 & 0,025 \\ \hline
	ab Platz 1001 &  21,500 & 0,000 \\ \hline
\end{tabular}
\end{center}

\textit{Beispiel: Platz 4 bekommt also 98,5 – 0,5 = 98,00 Punkte und Platz 5 dementsprechend 97,5 Punkte. Platz 12 sind 94,5 (94,75 – 0,25). Ab Platz 1001 bekämen alle weiteren platzierten Starter 21,5 Punkte.}

Für alle KIC und SIC Veranstaltungen werden die Punkte nach Platzierung vergeben. BIC Veranstaltungen, die auch so (Punkte nach Platzierung) gewertet werden, sind in Anhang~\ref{sec:appendix-veranstaltungen-nach-platzierung} zu finden.

\section{notwendige Ergebnisse für die Gesamtwertung}
\label{sec:appendix-notwendige-ergebnisse}
\subsection{BIC}
\label{subsec:bic-notwendige-ergebnisse}
\begin{center}
\begin{tabular}{|c|c|c|c|}
	\hline
	\thead{Durchgeführte\\Veranstaltungen} &
	\thead{Maximal\\gewertete\\Ergebnisse} &
	\thead{Minimal\\notwendige\\Ergebnisse} &
	\thead{Mögliche\\Streichergebnisse} \\
	\hline
	3 & 3 & \multirow{8}{*}{3} & 0 \\
	4 & 3 && 1 \\
	5 & 4 && 1 \\
	6 & 5 && 1 \\
	7 & 5 && 2 \\
	8 & 5 && 3 \\
	9 & 6 && 3 \\
	10 & 6 && 4 \\
	\hline
\end{tabular}
\end{center}

\subsection{KIC}
\begin{center}
\begin{tabular}{|c|c|c|c|}
	\hline
	\thead{Durchgeführte\\Veranstaltungen} &
	\thead{Maximal\\gewertete\\Ergebnisse} &
	\thead{Minimal\\notwendige\\Ergebnisse} &
	\thead{Mögliche\\Streichergebnisse} \\
	\hline
	3 & 3 & \multirow{8}{*}{3} & 0 \\
	4 & 3 && 1 \\
	5 & 4 && 1 \\
	6 & 5 && 1 \\
	7 & 5 && 2 \\
	8 & 5 && 3 \\
	9 & 6 && 3 \\
	10 & 6 && 4 \\
	\hline
\end{tabular}
\end{center}

\subsection{SIC}
\begin{center}
\begin{tabular}{|c|c|c|c|}
	\hline
	\thead{Durchgeführte\\Veranstaltungen} &
	\thead{Maximal\\gewertete\\Ergebnisse} &
	\thead{Minimal\\notwendige\\Ergebnisse} &
	\thead{Mögliche\\Streichergebnisse} \\
	\hline
	 3 & 3 & \multirow{8}{*}{3} & 0 \\
	 4 & 3 && 1 \\
	 5 & 4 && 1 \\
	 6 & 5 && 1 \\
	 7 & 5 && 2 \\
	 8 & 5 && 3 \\
	 9 & 6 && 3 \\
	10 & 6 && 4 \\
	\hline
\end{tabular}
\end{center}

\section{BIC Veranstaltungen mit Punktevergabe nach Platzierung}
\label{sec:appendix-veranstaltungen-nach-platzierung}
In der Saison 2024 werden für die folgenden BIC Veranstaltungen die Punkte nach Platzierung vergeben (siehe Anhang~\ref{sec:appendix-punkte-nach-platz}) und zusätzlich werden noch mögliche Bonuspunkte addiert (siehe Anhang~\ref{sec:appendix-punkte-nach-zeit}):

\begin{itemize}
	\item Nürnberg
	\item Bayreuth (1.06. - 2.06.)
	\item Leutershausen
	\item Bayreuth (14.09)
\end{itemize}
