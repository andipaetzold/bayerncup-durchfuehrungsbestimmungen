% !TeX spellcheck = de_DE
\section{Gesamtwertung der Veranstaltungen}
Die Einzelwertungen der Veranstaltungen werden zu einer Gesamtwertung für jeden Starter sowie für jedes Team zusammengerechnet. Wer die meisten Punkte hat, ist Sieger.

Die Gesamtwertung für jeden Starter erfolgt getrennt nach Damen und Herren:
\begin{itemize}
	\item BIC Gesamtwertung Overall
	\item BIC Gesamtwertung Overall U23 (17 bis 22 Jahre)
	\item BIC Altersklassenwertung:
	\begin{itemize}
		\item Junioren
		\item Aktive
		\item Senioren 35
		\item Senioren 40
		\item Senioren 50
		\item Senioren 60
		\item Senioren 70
	\end{itemize}
	\item KIC Altersklassenwertung:
	\begin{itemize}
		\item Schüler D
		\item Schüler C
		\item Schüler B
		\item Schüler A
		\item Cadetten
		\item Jugend
	\end{itemize}
	\item SIC
\end{itemize}

Die Gesamtwertung für jedes Team:
\begin{itemize}
	\item Teamwertung Nachwuchs
	\item TEAMS on Skates
\end{itemize}

Für Gesamtwertungen des BIC und SIC werden nur Sportler berücksichtigt, die sich zu Beginn der Saison für den jeweiligen Cup angemeldet haben.

\subsection{Gesamtwertung für jeden Starter}

Die Einzelwertungen werden in einer Gesamtwertung zusammengerechnet, wenn die minimal notwendige Anzahl von Ergebnissen bei Veranstaltungen erreicht wurde. Es wird nur eine maximale Anzahl der besten Ergebnisse in die Gesamtwertung berücksichtigt. Die restlichen Ergebnisse sind Streichresultate. Die maximale Anzahl der gewerteten Ergebnisse und die Streichresultate sind abhängig von der Anzahl der angebotenen Veranstaltungen (siehe Anhang~\ref{sec:appendix-notwendige-ergebnisse}).

\subsubsection{BIC Gesamtwertung Overall}
\label{subsec:bic-gesamtwertung-overall}
Es wird eine bestimmte Anzahl der besten Ergebnisse der BIC-Einzelwertungen zusammengerechnet, wenn minimal notwendige Anzahl von Ergebnissen erreicht wurde (Siehe Anhang~\ref{sec:appendix-notwendige-ergebnisse}).

\subsubsection{BIC Gesamtwertung Overall U23}
\label{subsec:bic-gesamtwertung-u23}
Wie~\ref{subsec:bic-gesamtwertung-overall}, jedoch nur für Starter, die 17 bis 22 Jahre alt sind. Für die Einteilung in diese Altersklasse ist jeweils das am 31. Dezember des laufenden Jahres erreichte Alter maßgeblich.

\subsubsection{BIC Altersklassenwertung}
\label{subsec:bic-ak-gesamtwertung}
Wie~\ref{subsec:bic-gesamtwertung-overall}, aber pro Altersklasse.

\subsubsection{KIC Altersklassenwertung}
\label{subsec:kic-gesamtwertung}
Es wird eine bestimmte Anzahl der besten Ergebnisse der KIC-Einzelwertungen zusammengerechnet, wenn minimal notwendige Anzahl von Ergebnissen erreicht wurde (siehe Anhang~\ref{sec:appendix-notwendige-ergebnisse}).

\subsubsection{SIC Gesamtwertung}
\label{subsec:sic-gesamtwertung}
Es wird eine bestimmte Anzahl der besten Ergebnisse der SIC-Einzelwertungen zusammengerechnet, wenn minimal notwendige Anzahl von Ergebnissen erreicht wurde (siehe Anhang~\ref{sec:appendix-notwendige-ergebnisse}).

\subsection{Teamwertungen}

\subsubsection{Teamwertung Nachwuchs}
\label{subsec:teamwertung-nachwuchs}
Die Punkte der KIC Altersklassenwertung von den besten drei Sportlern eines Teams werden addiert. 

Jedes Team hat die Möglichkeit bis zu 2 Teams in der Teamwertung Nachwuchs werten zu lassen.

\subsubsection{TEAMS on Skates}
\label{subsec:teams-on-skates}
Siehe Punkt 5 des Regelwerks TEAMS on Skates
