% !TeX spellcheck = de_DE
\section{Einzelwertung einer Veranstaltung}
Einspruch gegen die Gesamtwertung aufgrund einer falschen Punktzahl oder falschen Platzierung kann bis zu fünf Tagen vor dem Finalrennen eingereicht werden. Für Wettkämpfe innerhalb dieses Zeitraums entfällt die Einspruchszeit.

Für die Einzelwertung einer Veranstaltung können die Punkte nach Platzierung oder nach Zeit plus Bonuspunkte vergeben werden.

\subsection{BIC Wertung}
Für den BIC werden die Punkte grundsätzlich nach Zeit plus Bonuspunkte vergeben (siehe Anhang~\ref{sec:appendix-punkte-nach-zeit}). Eine Punktevergabe nach Platzierung oder eine Punktevergabe nach Platzierung plus Bonuspunkte ist auch möglich (siehe Anhang~\ref{sec:appendix-punkte-nach-platz}), wenn dies von Sportkommission Inline-, Fitness- und Speedskating beschlossen wurde (siehe Anhang~\ref{sec:appendix-veranstaltungen-nach-platzierung}). Bei einem Start aus mehreren Startblöcken heraus regelt die Ausschreibung bzw. die Information des Oberschiedsrichters vor dem Start, für welche Startblöcke Punkte vergeben werden. Die Kategorie “Fitness”, die durch einige Veranstalter angeboten wird, wird für die BIC Wertung nicht berücksichtigt.

\subsubsection{BIC AK Wertung}
Für die BIC Altersklassenwertung werden die Punkte aus der BIC Wertung verwendet.

\subsubsection{BIC Teamwertung bei einer BIC Veranstaltung (kein BIC Teamwettbewerb)}
Für die BIC Teamwertung werden die Punkte ausschließlich nach Platzierung vergeben (siehe Anhang~\ref{sec:appendix-punkte-nach-platz}). Für die Rangfolge zählen die drei punktbesten Sportler eines Teams (geschlechtsunabhängig).

\subsection{KIC Wertung}
Für die KIC Wertung werden die Punkte ausschließlich nach Platzierung vergeben (siehe Anhang~\ref{sec:appendix-punkte-nach-platz}). Werden mehrere Rennen an einer Veranstaltung durchgeführt, werden die Platzierungen der Rennen addiert und nach deren Summe von der kleinsten zur größten sortiert, wobei die kleinste Summe die beste Platzierung in der Wertung ergibt (= Platz 1). Bei Gleichstand, d.h. die Summe der Platzierungen in den Rennen ist gleich, regelt die jeweilige Ausschreibung, welcher Sportler besser platziert wird.

\subsection{SIC Wertung}
Für den SIC werden die Punkte ausschließlich nach Platzierung vergeben (siehe Anhang~\ref{sec:appendix-punkte-nach-platz}).