% !TeX spellcheck = de_DE
\section{Startberechtigung}
Grundsätzlich ist in allen Rennen jedermann startberechtigt. Eine Lizenz des DRIV oder eine vom DRIV anerkannte Lizenz ist nicht notwendig. Einschränkungen der Startberechtigung ergeben sich aus der Altersklasse des Starters (siehe Anhang~\ref{sec:appendix-altersklassen}) sowie der Kategorie des Rennens.

Ein Hochstarten der Schüleraltersklassen (Schüler D bis Schüler B), ist nicht erlaubt. Die Altersklassen Schüler A, Cadetten, Jugend und Junioren dürfen in eine ältere Altersklasse oder bis in die Aktivenklasse hochstarten soweit es keine Beschränkung durch die maximale Streckenlänge gibt (siehe Anhang~\ref{sec:appendix-altersklassen}). Die Seniorenaltersklassen dürfen in eine jüngere oder bis in die Aktivenklasse hochstarten.
Es besteht die Möglichkeit, dass Sportler während einer gesamten Saison hochstarten. Die Altersklasse eines Sportlers wird mit der Wertung seines ersten Wertungslaufes festgelegt. Ein nachträgliches Ändern ist nicht mehr möglich. Sollte ein Sportler während einer laufenden Saison in einer anderen Altersklasse an den Start gehen, so wird das Ergebnis in der Gesamtwertung nicht berücksichtigt. Bietet ein Veranstalter abweichende Altersklassen zur nachfolgenden Angabe an, so gilt dies nicht als Hochstart. Die Sportler werden im Nachhinein den entsprechenden Altersklassen zugeordnet und bei einer BIC Altersklassenwertung bzw. KIC Wertung auch neu platziert.

\subsection{Startberechtigung BIC}
Die folgenden Altersklassen sind startberechtigt:
\begin{itemize}
	\item Junioren
	\item Aktive
	\item Senioren 30
	\item Senioren 40
	\item Senioren 50
	\item Senioren 60
	\item Senioren 70
\end{itemize}

\subsection{Startberechtigung KIC}
Die folgenden Altersklassen sind startberechtigt:
\begin{itemize}
	\item Schüler D
	\item Schüler C
	\item Schüler B
	\item Schüler A
	\item Cadetten
	\item Jugend
\end{itemize}

\subsection{Startberechtigung SIC}
Die folgenden Altersklassen sind startberechtigt:
\begin{itemize}
	\item Schüler A
	\item Cadetten
	\item Jugend
	\item Junioren
	\item Aktive
	\item Senioren 30
	\item Senioren 40
	\item Senioren 50
	\item Senioren 60
	\item Senioren 70
\end{itemize}